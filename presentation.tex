\documentclass[14pt, aspectratio=169]{beamer}
\usepackage[utf8]{inputenc}
\usetheme{default}
\useoutertheme{infolines}
\setbeamertemplate{navigation symbols}{}
% \usecolortheme{owl}
\setbeamertemplate{headline}{}

\title{Problems Faced in Programming for Research}
\subtitle{Causes, Effects and Some Strategies}
\author{Cristian Dinu}
\date{March 2023}
\begin{document}

\frame{\titlepage}

\begin{frame}
\frametitle{Introduction}
\begin{itemize}
\item Scientific programming is still programming
\item The main issues of general programming still apply
    \begin{itemize}
    \item Unclear requirements
    \item Tough stakeholder management
    \item Bad planning
    \end{itemize}
\item But I will talk about problems specific to programming for research
\end{itemize}

\end{frame}

\begin{frame}
\frametitle{1. Both research and coding are hard}
So that one either can do well one or the other
\begin{itemize}
\item Effects: It is challenging to collect the requirement. Misunderstandings and friction happen between the researcher and the coder.
\item Strategies: Iterate fast, show and validate early, using friendly formats (Jupyter, Excel), avoid computer/programming slang
\end{itemize}
\end{frame}


\begin{frame}
\frametitle{2. The "Paper" is more important than the code}
The incentives and priorities seem often wrong. Recognition, too.
\begin{itemize}
\item Effects: Quick and dirty duct-tape coding, neglected documentation, testing, maintainability, and reproducibility
\item Strategies: Recognise software contributions, emphasize code quality, make code as important as the paper
\end{itemize}
\end{frame}

\begin{frame}
\frametitle{3. People who code seem "in transit" to better roles or self-taught\ldots}
I saw most of the code is done "for free" by masters or PhD students or enthusiastic self-taught young academics.
\newline
\begin{itemize}
    \item Effects: No good practices, inconsistent code quality, slow skills improvement
    \item Strategies: Educate beyond coding -- e.g. source control, documentation, and establish some standards (e.g. everything goes on departments' GitHub account, no Excel allowed)
\end{itemize}
\end{frame}

\begin{frame}
\frametitle{4. \ldots because good developers are expensive and industry pays well}
Also, it's unfair to pay a developer three times as much as a researcher, just because the IT pays well.
\begin{itemize}
\item Effects: Difficulty attracting and retaining skilled engineers, reliance on inexperienced developers
\item Strategies: Allow people to be paid directly from grants, or projects with the industry. 
\end{itemize}
\end{frame}

\begin{frame}
\frametitle{More and more blackboxes (libraries, AI models)}
\begin{itemize}
\item Effects: Reduced transparency, harder debugging, validation issues, over-reliance on external libraries
\item Strategies: Encourage open-source development, require explainable AI, prioritize validation and testing
\end{itemize}
\end{frame}

\begin{frame}
\frametitle{Conclusions}
\begin{itemize}
\item Programming for research shares difficult problems with "general" programming 
\item Competition with the industry is unfair [compared with academia]
\item Hopefully a "software is an investment not an expense" mindset will be adopted by the majority of people involved in research
\end{itemize}
\end{frame}

\end{document}
